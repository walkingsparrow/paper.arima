\documentclass[english,12pt]{article}
\usepackage[T1]{fontenc}
\usepackage[latin9]{inputenc}
\usepackage{float}
\usepackage{graphicx}
\usepackage{esint}
\usepackage{natbib}
\usepackage{times,bm}
\usepackage{babel}

\makeatletter
\linespread{1.25}
\pagestyle{headings}

\begin{document}

\title{Implementation of Parallel Algorithms for ARIMA in Distributed
  Database Systems}

% \author{}

\maketitle

\begin{abstract} MADlib is an open-source library for scalable in-database
  analytics. We impemented ARIMA in MADlib's framework. The algorithms for
  fitting ARIMA model to time series are intrinsically sequential, because any
  calculation for a specific time $t$ depends on the result from the
  calculation for $t-1$.  This makes it difficult to parallelize the ARIMA
  model fitting. Our solution parallelizes the computation by splitting the
  data into $n$ chunks. Since the model fitting involves multiple iterations of
  computations. We use the results from previous iteration as the initial
  values for each chunk. Thus the computation for each chunk of data does not
  need to wait for the results from previous chunk. Another technique is used
  to improve the performance of the implementation, which redistributes the
  original data so that each chunk can be completely loaded into memory.
  \end{abstract}

\section{Introduction}

% Introduce to MADlib, and implement ARIMA in MADLib
MADlib is an open source library for scalable in-database analytics. It is
created by the Predictive Analytics Team at Pivotal Inc. MADlib provides
data-parallel implementations of mathematical, statistical and machine-learning
methods for structured and unstructured data. 

\section{MADlib}

% More details on MADlib

\section{Implementation of ARIMA}

% A short summary for the next few subsections

\subsection{The Algorithm}

% List the LM algorithm

\subsection{The Problems in Parallelization}

% What are the difficulties that we are facing

\subsection{Our Solution}

% Our solution and how to improve the performance

\subsection{Performance}

% Good performance improvement, some examples

\subsection{Discussion}

% Can be applied to other algorithms

\section{Conclusion}

% \bibliographystyle{apalike}
% \bibliography{financial_market}

\end{document}
