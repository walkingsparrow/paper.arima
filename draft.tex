\documentclass[english,12pt]{article}
\usepackage{babel}
\usepackage{natbib}
\usepackage[latin1]{inputenc}
\usepackage[T1]{fontenc}
\usepackage{amsmath}
\usepackage{amssymb}
\usepackage{algorithm2e}

%\makeatletter
\linespread{1.25}

% \pagestyle{headings}

\begin{document}

\title{Implementation of Parallel Algorithms for ARIMA in Distributed
  Database Systems}

% \author{}

\maketitle

\begin{abstract}
    MADlib is an open-source library for scalable in-database analytics. We
    impemented ARIMA in MADlib's framework. The algorithms for fitting ARIMA
    model to time series are intrinsically sequential, because any calculation
    for a specific time $t$ depends on the result from the calculation for
    $t-1$.  This makes it difficult to parallelize the ARIMA model fitting. Our
    solution parallelizes the computation by splitting the data into $n$
    chunks. Since the model fitting involves multiple iterations of
    computations. We use the results from previous iteration as the initial
    values for each chunk. Thus the computation for each chunk of data does not
    need to wait for the results from previous chunk. Another technique is used
    to improve the performance of the implementation, which redistributes the
    original data so that each chunk can be completely loaded into memory.
\end{abstract}

%\maketitle

\section{Introduction}

% Introduce to MADlib, and implement ARIMA in MADlib
MADlib~\cite{madlib} is an open source library for scalable
in-database analytics, currently backed by the Predictive Analytics
Team at Pivotal Inc. It provides data-parallel implementations of
mathematical, statistical and machine-learning methods for structured
and unstructured data.

Time series analyses are important in econometrics, mathematical
finance, weather forecasting, earthquake prediction and many other
fields. For example, one of the most conspicuous data analytics task
-- stock price forecasting -- falls into the category of time series
analysis.

However, different from other common data analytics tasks, time series
data have a natural temporal ordering. And many time series modeling
methods, such as ARIMA~\cite{arima} and Cox proportional
hazards~\cite{cox} therefore depend on sequential processing of time
series data, which raises a challenge for the data-parallel
implementation in MADlib.

In MADlib, we attack this challenge using an idea illustrated below
[TODO: add a simple figure to compare MADlib ARIMA and order aggregate
with iterations].


\subsection{What is MADlib?}

% More details on MADlib

MADlib is created by the Predictive Analytics Team at Pivotal Inc. (previously
Greenplum). It is an add-on package for Greenplum database or PostgreSQL
database. The package itself is open sourced.

Cohen et al.~\cite{mad-skills} explained that ``MAD'' stands for ``magnetic'',
``agile'', ``deep'', and ``lib'' stands for advanced (mathematical, statistical,
machine learning), parallel and scalable in-database functions.

The MADlib project was initiated in late 2010. Currently MADlib's latest
version is 1.3. By itself MADlib provides a pure SQL interface. An R
front-end named PivotalR~\cite{pivotalr} is also created by the
Predictive Analytics Team at Pivotal Inc.

%A Python wrapper~\cite{python-madlib} for MADlib is also available.

MADlib can be installed and run on both Grenplum database and PostgreSQL\@. On
Greenplum database (GPDB) systems, MADlib utilizes the data parallel
funtionality. The calculation is done on multiple segments of GPDB in parallel,
and the results from the segments are summarized on the master node. In many
cases, multiple iterations of such calculations are needed. We use C++ code to
implement the part inside each iteration, which is computation intensive. We
use Python code to drive the iterations, which does not have high requirements
for the performance.

Although MADlib does not do parallel computation on PostgreSQL, it is
still valuable for process big data sets that cannot be loaded into
memory. For example, in this paper we describe our implementation of
ARIMA in MADlib. We compare our implementation 

% explain more about the C++ abstraction layer

% more about Python layer

At the time of writing, MADlib has modules for linear regression,
logistic regression, multinomial logistic regression, elastic-net
regularization for linear and logistic regressions, all kinds of
robust estimators for regressions, marginal effects, k-means,
association rules, cross validation, linear systems, matrix
factorization, LDA, data summary, correlation, hypothesis tests, SVD
and PCA, ARIMA etc and many other supporting functions.  A very
detailed user documentation is available onlin at http://madlib.net.

Here in this paper, we focus our implementation for ARIMA in MADlib.

\section{Implementation of ARIMA}

% A short summary for the next few subsections

In the next few subsections, we will first describe the algorithm that we have
used to solve the problem of maximization of partial log-likelihood. Then, we
point out the reason why it is difficult to make the algorithm run parallel.
The difficulty is not restricted to this specific algorithm. It is a
common situation in all algorithms that fit ARIMA models. Actually it is easy
to see that many machine learning models have similar problems. Next, we
describe our solution to this problem. Our method is quite general and can
parallelize similar problems. Then we describe a simple method to improve the
performance of our algorithm. In the end, we discuss the possible ways to
generalize our methods to more algorithms.

\subsection{The Algorithm}

% List the LM algorithm

An ARIMA model is an auto-regressive integrated moving average model. An ARIMA
model is typically expressed in the form
\begin{equation}
(1 - \phi(B)) Y_t  = (1 + \theta(B)) Z_t,
\end{equation}
where $B$ is the backshift operator. The time $t$ is from $1$ to $N$.

ARIMA models involve the following variables:
\begin{enumerate}
   \item The lag difference $Y_{t}$, where  $Y_{t} = {(1-B)}^{d}(X_{t} - \mu)$.
    \item The values of the time series $X_t$.
    \item $p$, $q$, and $d$ are the parameters of the ARIMA model.
      $d$ is the differencing order, $p$ is the order of the AR
      operator, and $q$ is the order of the MA operator.
    \item The AR operator $\phi(B)$.
    \item The MA operator $\theta(B)$.
    \item The mean value $\mu$, which is always set to be zero for
      $d>0$ or need to be estimated.
    \item The error terms $Z_t$.
\end{enumerate}

The  auto regression operator models the prediction for the next
observation  as some linear combination of the previous observations.
More formally, an AR operator of order $p$ is defined as
\begin{equation}
\phi(B) Y_t= \phi_1 Y_{t-1}   + \cdots +  \phi_{p} Y_{t-p}
\end{equation}

The moving average operator is similar, and it models the prediction
for the next observation as a linear combination of the errors in the
previous prediction errors.  More formally, the MA operator of order
$q$ is defined as
\begin{equation}
\theta(B) Z_t =   \theta_{1} Z_{t-1} + \cdots + \theta_{q} Z_{t-q}.
\end{equation}

We assume that
\begin{equation}
\Pr(Z_t) = \frac{1}{\sqrt{2 \pi \sigma^2}} e^{-Z^2_t/2 \sigma^2}, \quad t > 0
\end{equation}
and that  $Z_{-q+1} = Z_{-q+2} = \cdots = Z_0 = Z_1 = \cdots = Z_p =
0$. The initial values of $Y_t=X_t-\mu$ for $t=-p+1, -p+2, \dots,
0$ can be solved from the following linear equations
\begin{eqnarray}
\phi_1 Y_0 + \phi_2 Y_{-1} + \cdots + \phi_p Y_{-p+1} &=& Y_1 \nonumber\\
\phi_2 Y_0 + \cdots + \phi_p Y_{-p+2} &=& Y_2 - \phi_1 Y_1  \nonumber\\
&\vdots& \nonumber\\
\phi_{p-1} Y_0 + \phi_p Y_{-1} &=& Y_{p-1} - \phi_1 Y_{p-2} - \cdots -
\phi_{p-2} Y_1 \nonumber \\
\phi_p Y_0  &=& Y_p - \phi_1 Y_{p-1} - \cdots - \phi_{p-1} Y_{1} \label{eq:init_Y}
\end{eqnarray}

The likelihood function $L$ for $N$ values of $Z_t$  is then
\begin{equation}
L(\phi, \theta) = \prod_{t = 1}^N  \frac{1}{\sqrt{2 \pi \sigma^2}} e^{-Z^2_t/2 \sigma^2}
\end{equation}
so the log likelihood function $l$ is
\begin{eqnarray}
l(\phi, \theta) &=& \sum_{t = 1}^N \ln \left(\frac{1}{\sqrt{2 \pi \sigma^2}}
 e^{-Z^2_t/2 \sigma^2}
 \right) \nonumber\\
 &=&  \sum_{t = 1}^N  - \ln \left( \sqrt{2 \pi \sigma^2}\right)
-\frac{Z^2_t}{2 \sigma^2}\nonumber\\
&=&  -\frac{N}{2} \ln \left( 2 \pi \sigma^2\right)  - \frac{1}{2
  \sigma^2} \sum_{t = 1}^N   Z^2_t\ \. \label{eq:loglikelihood}
\end{eqnarray}
Thus, finding the maximum likelihood is equivalent to solving the
optimization problem (known as the conditional least squares
formation)
\begin{equation}
\min_{\theta, \phi} \sum_{t = 1}^N  Z^2_t.
\end{equation}
The error term $Z_t$ can be computed iteratively as follows:
\begin{equation}
    Z_t = X_t - F_t(\phi, \theta, \mu) \label{eq:error-terms}
\end{equation}
where
\begin{equation}
F_t(\phi, \theta, \mu) = \mu + \sum_{i=1}^p \phi_i (X_{t-i}-\mu) + \sum_{i=1}^q
\theta_i Z_{t-i}
\end{equation}

In mathematics and computing, the Levenberg-Marquardt algorithm (LMA),
also known as the damped least-squares (DLS) method, provides a
numerical solution to the problem of minimizing a function, generally
nonlinear, over a space of parameters of the function. These
minimization problems arise especially in least squares curve fitting
and nonlinear programming.

To understand the Levenberg-Marquardt algorithm, it helps to know the
gradient descent method and the Gauss-Newton method.  On many
``reasonable'' functions, the gradient descent method takes large
steps when the current iterate is distant from the true solution, but
is slow to converge an the current iterate nears the true solution.
The Gauss-Newton method is much faster for converging when the current
iterate is in the neighborhood of the true solution.  The
Levenberg-Marquardt algorithm tries to get the best of best worlds,
and combine the gradient descent step with Gauss-Newton step in a
weighted average.  For iterates far from the true solution, the step
favors the gradient descent step, but as the iterate approaches the
true solution, the Gauss-Newton step dominates.

Like other numeric minimization algorithms, LMA is an iterative
procedure.  To start a minimization, the user has to provide an
initial guess for the parameter vector, $p$, as well as some tuning
parameters $\tau$, $\epsilon_1$, $\epsilon_2$, $\epsilon_3$, and $k_{\max}$.
Let $Z(p)$ be the vector of calculated errors ($Z_t$'s) for the
parameter vector $p$, and let $J = {(J_{1}, J_{2}, \ldots, J_N)}^T$
be a Jacobian matrix.

A proposed implementation is as follows:

\begin{algorithm}[ht]
\KwIn{An initial guess for parameters $\vec{\phi}_0, \vec{\theta}_0, \mu_0$}
\KwOut{The parameters that maximize the likelihood $\vec{\phi}^*,
  \vec{\theta}^*, \mu^*$}
    $k \leftarrow 0$; $v \leftarrow 2$;
    $(\vec{\phi},\vec{\theta},\mu) \leftarrow
    (\vec{\phi}_0,\vec{\theta}_0,\mu_0)$\;
    Calculate $Z(\vec{\phi},\vec{\theta},\mu)$ with
    Eq. (9).  \textit{\footnotesize \# Vector of errors}\;
    $A \leftarrow J^T J$   \textit{\footnotesize \# The Gauss-Newton Hessian
    approximation}\;
    $u \leftarrow \tau * \max_i(A_{ii})$ \textit{\footnotesize \# Weight of the
    gradient-descent step}\;
    $g \leftarrow J^T Z(\vec{\phi},\vec{\theta},\mu)$ \textit{\footnotesize \#
    The gradient descent step}\;
	$ \text{stop} \leftarrow (\|g\|_{\infty} \le \epsilon_1)$
    \textit{\footnotesize \# Termination Variable}\;
	\While{not stop and $k < k_{\max}$}{
		$k \leftarrow k + 1$\;
		\Repeat{stop or $\rho > 0$}{
            $\delta \leftarrow{(A + u \times \text{diag}(A))}^{-1} g$
            \textit{\footnotesize \# Calculate step direction}\;
            \eIf(\textit{\footnotesize \# Change  is too small
              to
              continue.})
            {$\| \delta \| \le \epsilon_2 \|
              (\vec{\phi},\vec{\theta},\mu) \|$} {
				$\text{stop} \leftarrow \text{true}$\;
			}{
				$(\vec{\phi}_{new},\vec{\theta}_{new},\mu_{new})
                \leftarrow (\vec{\phi},\vec{\theta},\mu) + \delta$
                \textit{\footnotesize \# A trial step}\;
				$\rho \leftarrow (\|
                Z(\vec{\phi},\vec{\theta},\mu)\|^2 - \|
                Z(\vec{\phi}_{new},\vec{\theta}_{new},\mu_{new})\|^2
                )/(\delta^T(u \delta + g))$ \;
                \eIf(\textit{\footnotesize \# Trial step was good}){$\rho >
                    0$}{
					$(\vec{\phi},\vec{\theta},\mu) \leftarrow
                    (\vec{\phi}_{new},\vec{\theta}_{new},\mu_{new})$
                    \textit{\footnotesize \# Update variables}\;
                    Calculate $Z(\vec{\phi},\vec{\theta},\mu)$ with
                    Eq. (9); $A \leftarrow J^T J$; $g \leftarrow J^T
                    Z(\vec{\phi},\vec{\theta},\mu)$\;
					$ \text{stop} \leftarrow (\|g\|_{\infty} \le
                    \epsilon_1)$ or $(\|
                    Z{(\vec{\phi},\vec{\theta},\mu)}^2 \| \le
                    \epsilon_3)$ \;
                    $v \leftarrow 2$;
					$u \rightarrow u * \max(1/3, 1 - {(2\rho - 1)}^3 )$\;
                } (\textit{\footnotesize \# Trial step was bad}){
                    $v \leftarrow 2 v$;
					$u \leftarrow u v$\;
				}
			}
		 }
	}
	$(\vec{\phi}^*,\vec{\theta}^*,\mu^*) \leftarrow (\vec{\phi},\vec{\theta},\mu)$\;

\label{alg:LM}
\end{algorithm}

Suggested values for the tuning parameters are $\epsilon_1 = \epsilon_2 =
\epsilon_3 = 10^{-15}, \tau = 10^{-3}$ and $k_{\max} = 100$.

The Jacobian matrix $J = {(J_{1}, J_{2}, \ldots, J_N)}^T$ requires the partial derivatives, which are
\begin{equation}
J_t = {(J_{t, \phi_1}, \ldots, J_{t,\phi_p}, J_{t,\theta_1}, \ldots,
  J_{t,\theta_q}, J_{t,\mu})}^T
\end{equation}
Here the last term is present only when \texttt{include\_mean} is
\texttt{True}.
The iteration relations for $J$ are
\begin{eqnarray}
J_{t, \phi_i} &=& \frac{\partial F_t(\phi,\theta)}{\partial \phi_i} =
-\frac{\partial Z_t}{\partial \phi_i} = X_{t-i}-\mu + \sum_{j=1}^q
\theta_j \frac{\partial Z_{t - j}}{\partial \phi_i} = X_{t-i}-\mu - \sum_{j=1}^q
\theta_j J_{t-j,\phi_i}, \label{eq:J1}\\
J_{t, \theta_i}&=&\frac{\partial F_t(\phi,\theta)}{\partial \theta_i} =
-\frac{\partial Z_t}{\partial \theta_i} = Z_{t-i} + \sum_{j =1}^q
\theta_j \frac{\partial Z_{t - j}}{\partial \theta_i} = Z_{t-i} -
\sum_{j=1}^q \theta_j J_{t-j,\theta_i}, \label{eq:J2}\\
J_{t, \mu} &=&\frac{\partial F_t(\phi,\theta)}{\partial \mu} =
-\frac{\partial Z_t}{\partial \mu} = 1 -
\sum_{j=1}^p \phi_j - \sum_{j=1}^q \theta_j \frac{\partial
  Z_{t-j}}{\partial \mu} = 1 - \sum_{j=1}^p \phi_j - \sum_{j=1}^q
\theta_j J_{t-j,\mu}. \label{eq:J3}
\end{eqnarray}
Note that the mean value $\mu$ is considered separately in the above
formulations. When \texttt{include\_mean} is set to \texttt{False}, $\mu$ will be simply
set to 0. Otherwise, $\mu$ will also be estimated together with
$\vec{\phi}$ and $\vec{\theta}$. The initial conditions for the above
equations are
\begin{equation}
J_{t,\phi_i} = J_{t,\theta_j} = J_{t,\mu} = 0 \quad \mbox{for }
t \leq p,\ \mbox{and }i=1,\dots,p; j = 1, \dots, q\ ,
\end{equation}
because we have fixed $Z_t$ for $t\leq p$ to be a constant $0$ in the initial
condition. Note that $J$ is zero not only for $t\leq
0$ but also for $t\leq p$.

\subsection{The Problems in Parallelization}

It is very easy to see that Eqs.
(\ref{eq:error-terms},~\ref{eq:J1},~\ref{eq:J2},~\ref{eq:J3}) are
difficult to parallelize. Each step computation uses the result from the
previous step. Therefore, we have to scan through the data sequentially to
compute the quatities in these equations.

\subsection{Our Solution}

% Our solution and how to improve the performance

In order to utilize the data-parallel capability of MPP database, we propose to segment the whole time series data into a set of consecutive time series data, numbering from $1$ to $N$, each subset comprising a sequence of consecutive time series data of size $K$. Since the model fitting involves multile iterations of computations, for any subset $i$ execpt for the first one, we use the results of the previous subset $i-1$ from the previous iteration as the initial values. Thus in each iteration, the computation for each subset of data does not need to wait for the results from the previous subset. In this way, the model fitting computaion can be parallelized. Besides, we find that aggregating each subset of consecutive time series data into an array (i.e. a data chunk) can greatly simpliy the implementation (for example, from a stateful window function implementation to a plain UDF implementation) and accelerate the computation (mainly due to the reduction of I/O overhead and function call overhead). Furthermore, we can specify the value of size $K$  according to the system configuration so that each chunk of data can be completely loaded into the avaialble memory).

The following SQL shows how we segment and redistribute the data according to the above proposals. 

\begin{verbatim}
create temp table dist_table as
  select
    ((tid - 1) / chunk_size)::integer + 1 as distid, 
    tid, tval
  from 
      input_table
distributed by (distid)
\end{verbatim}

\subsection{Performance}

% Good performance improvement, some examples

\subsection{Discussion}

% Can be applied to other algorithms

\section{Conclusion}

\bibliographystyle{abbrv}

\bibliography{pred}

\end{document}
